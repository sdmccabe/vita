\documentclass[11pt, letter]{article}
\usepackage{jk-vita}
\usepackage{enumitem}
\usepackage{fontawesome5}

\title{}
\name{Stefan D. McCabe}
\address{
    805 21\textsuperscript{st} Street \textsc{NW}\\
    Unit 201 \\
    Washington, DC 20052
}
\www{sdmccabe.github.io}
\email{stefanmccabe@gmail.com}
\twitter{mccabe\_s}
\tel{(571) 432-8219}
\subject{}
\hypersetup{xetex,
    colorlinks=true,
    urlcolor=ImperialBlue,
    citecolor=ImperialBlue,
}

\begin{document}

\maketitle
\section{skills}
\textbf{Programming}\hfill Python (pandas, PySpark, matplotlib), R (tidyverse, ggplot2), SQL\\ \vspace{1mm}
\textbf{Data Management} \hfill Continuous integration, Apache Airflow, Unix command-line tools, git\\\vspace{1mm}
\textbf{Quantitative Methods} \hfill Regression, causal inference, exploratory data analysis \\\vspace{1mm}
\textbf{Machine Learning} \hfill Natural language processing, network analysis\\\vspace{1mm}
\textbf{Collaboration/Communication} \hfill Teaching, interdisciplinary communication, code review


\section{experience}
\textbf{Institute for Data, Democracy \& Politics, The George Washington University} \hfill 12/2022-- \\
Postdoctoral Associate (40 hours/week) \\
%Constructed data pipelines for studying social media data and applied machine-learning techniques to study public opinion and political communication.

\begin{itemize}[noitemsep,topsep=0pt]
\item[-] Constructed and maintained data pipelines for studying social media.
\item[-] Applied machine-learning techniques to study public opinion and political communication.
\end{itemize}

\vspace{2.75mm}
\textbf{Lazer Lab, Northeastern University} \hfill 09/2016--05/2020, 09/2020--11/2022 \\
Research Assistant (20 hours/week)\\
%Work at all levels (theorizing, data engineering, analysis, writing) to study politics online.
\begin{itemize}[noitemsep,topsep=0pt]
\item[-] Analyzed billions of Tweets connected to voter data to study political communication.
\item[-] Developed software to enable comparative analysis of graph distance measures. \\
\item[-] Processed terabytes of mobile-phone location events to study mobility during COVID-19. \\
\item[-] Presented research in various forms: reports to government officials, peer-reviewed journal articles, and conference presentations.
\item[-] Selected articles published in \href{https://academic.oup.com/poq/article/85/S1/323/6342443?guestAccessKey=3c4778c7-e064-4647-b223-6caa3ab9e002}{\emph{Public Opinion Quarterly}}, \href{https://www.pnas.org/doi/10.1073/pnas.2115900119}{\emph{Proceedings of the National Academy of Sciences}}, \href{https://royalsocietypublishing.org/doi/10.1098/rspa.2019.0744}{\emph{Proceedings of the Royal Society A}}, or available on the \href{https://arxiv.org/abs/2212.08873}{arXiv}.
\end{itemize}

\vspace{2.75mm}
\textbf{Microsoft Research NYC}  \hfill 05/2020--09/2020 \\
Research Intern (40 hours/week)\\
\begin{itemize}[noitemsep,topsep=0pt]
\item[-] Worked with mentor to study agenda-setting among Twitter-using journalists.
\end{itemize}

\vspace{-1mm}

\section{education}
\textbf{Northeastern University}  \hfill 2022 \\
PhD, Network Science %\\
%Dissertation: ``Essays on the Measurement of Online Behavior''
\vspace{2.75mm}

\textbf{George Mason University}  \hfill 2016 \\
MA, Computational Social Science %\\
%Thesis: ``Communicating Sequential Agents: An Analysis of Concurrent Agent Scheduling''
\vspace{2.75mm}

\textbf{George Mason University}  \hfill 2013 \\
BA, Government \& International Politics \\

\vspace{2mm}

% \section{selected projects}

% \textbf{Political Communication on Twitter} \\
% Analyzed billions of Tweets connected to voter data to study patterns of political communication. \\
% Work published in \href{https://academic.oup.com/poq/article/85/S1/323/6342443?guestAccessKey=3c4778c7-e064-4647-b223-6caa3ab9e002}{\emph{Public Opinion Quarterly}}, \href{https://journalqd.org/article/view/2570}{\emph{Journal of Quantitative Description: Digital Media}}.

% \vspace{2.75mm}
% \textbf{Big Data Analysis of Compliance With Social Distancing}\\
% Processed terabytes of mobile-phone location events to study mobility changes during COVID-19. \\
% Dashboard and reports \href{https://covid19.gleamproject.org/mobility}{available online}.

% \vspace{2.75mm}
% \textbf{Network Comparison} \\
% Developed software, \href{https://github.com/netsiphd/netrd}{\texttt{netrd}}, to enable comparative analysis of graph distance measures. \\
% Work published in \href{https://royalsocietypublishing.org/doi/10.1098/rspa.2019.0744}{\emph{Proceedings of the Royal Society A}}, \href{https://joss.theoj.org/papers/10.21105/joss.02990}{\emph{Journal of Open Source Software}}.

\section{teaching}
\textbf{Complex Networks and Applications} \\
Graduate course, \href{https://web.archive.org/web/20210102223738/barabasilab.com/course}{Fall 2020}. Survey course on network science.
\vspace{2.75mm}

\textbf{Programming with Data: Social Science Practicum} \\
Undergraduate course, \href{https://sdmccabe.github.io/ds2001/}{Fall 2019}. Hands-on instruction using Python and social-science examples.

\end{document}
